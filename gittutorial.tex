\documentclass{beamer}
%%%%%%%%%%%%%%%%%%%%%%%%%%%%%%%  Packages  %%%%%%%%%%%%%
\usepackage{amsmath} 
\usepackage{mathtools}
\usepackage{physics}
\usepackage{amssymb}
\usepackage{mathptmx}
\usepackage{array}
  
%%%%%%%%% FIGurES %%%%%%%%%%%%%%%%%%%%%%%%
\usepackage{textcomp}
\usepackage{graphicx}
\usepackage{caption} 
\usepackage{subcaption}
\usepackage{scrextend}
\usepackage{rotating}
\usepackage{float}
\usepackage{hyperref}
\hypersetup{colorlinks=true, citecolor=blue, linkcolor=blue}
\renewcommand{\equationautorefname}{Eq.}
\renewcommand{\figureautorefname}{Fig.}
 
%%%%%%%%%%%% LaNgUaGe %%%%%%%%%%%%%%%%%%
\usepackage{verbatim}
\usepackage{natbib}
%\usepackage{qcircuit}
\usepackage{wrapfig}

\usepackage[utf8]{inputenc}
\usetheme{PaloAlto}

\title{Version control using Git and Plotting Tutorial}
\author{Oliver Thomas}
\institute{Quantum Engineering CDT \\ University of Bristol}
\date{\today}

\begin{document}

\frame{\titlepage}

% slide 1
\begin{frame}
\frametitle{Why you should use version control}
\begin{itemize}
\item Collaborative work  
\item Working remotely
\end{itemize}
\end{frame}

%slide 2
\begin{frame}
\frametitle{What is Git?}
\begin{itemize}
\item  
\item
\end{itemize}
\end{frame}

%slide 3
\begin{frame}
\frametitle{Making a repository}
\begin{itemize}
\item You can do this online on the Github website   
\item Create a new repository
\item Then click clone to get the url, open git on your computer and type: 
	\texttt{git clone url}
\end{itemize}
\end{frame}

%slide 3
\begin{frame}
\frametitle{Making a repository}
\begin{itemize}
\item make a new repository, then go to the website and make a new folder 
\item go to the folder and right click git with bas
\item You are now able to use bash for the rest of the talk!
\end{itemize}
\end{frame}


%slide 4
\begin{frame}
\frametitle{Basic Git commands}
\begin{itemize}
	\item There are four\footnotemark important commands you will need for git:
	\item \texttt{git pull}
	\item \texttt{git add}
	\item \texttt{git commit}
	\item \texttt{git push}
\end{itemize}
\footnotetext[1]{I cheat here and write a bash script which does these in order so I only have to run 1 command.}
\end{frame}

%slide 5
\begin{frame}
\frametitle{Plotting with Python} 
\begin{itemize}
	\item Python is popular, multi-platform and becoming a standard language\footnotemark 
\end{itemize}


\footnotetext[2]{standard on nearly every linux dist}
\url{https://matplotlib.org/api/_as_gen/matplotlib.pyplot.plot.html}
\end{frame}

%slide 6
\begin{frame}
\frametitle{Plotting}
\end{frame}

\end{document}
