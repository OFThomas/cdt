\section{The Schur Transform}

\textit{background.}

2004 Bacon, Chuang \& Harrow proposed a scheme for implementing the Schur transform in Poly time. 


Information processing tasks, such as classical compression gain a huge advantage implementing the process using a streaming scheme. Rather than start the compression on all the data to be sent and wait for it all to be compressed then send the data, as compression can be performed sequentially, compress part of the message and send it while compressing the next part. The Schur transform can also be thought of in this way. However, with compression it can still be useful to only compress part of the message, it is meaningless to perform only part of the Schur transform which suggests there may be a more optimal scheme opposed to streaming for implementing the Schur transform.      

This report is structured as follows,   

There are two distinct ways of performing the Schur transform on $n$ qubits, it can either be built up from coupling
all $n$ qubits together in a single iteration which we call the spatial multiplexed approach. The other approach is 
performing Clebsch-Gordan (CG) transforms on the $n$ qubits one at a time which we call the temporal multiplexed 
approach.  



