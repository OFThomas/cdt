\appendix
\section{Appendix- Circuits}
%\vspace{-1cm}

For the general circuit structure, the M register is updated, Controlled rotation ($R_\theta$) is applied to the qubit and then the J register is updated.

The case where $\ket{S} = \ket{0}$ means the spin is $+\frac{1}{2}$ so to add $\frac{1}{2}$ to M, 1 is added to the $m_0$ qubit. The first Quantum Adder (QAdd) uses Toffoli gates controlled on $\ket{s}=\ket{0}$ (denoted by the white control circle) with the current $m_0$ value and $C_0$ (an ancilla carry). This ensures that the case when $m_0 = 1$ and 1 is added to it, $m_0$ goes to 0 and $m_1$ is increased using the carry as $001+1=010$. The rest of the QAdd stages then just check the carry of the previous qubit to complete to M $+ \frac{1}{2}$ addition as $\ket{S} = \ket{0}$ does not trigger any of the rest of the control gates.

The case where $\ket{S} = \ket{1}$ means the spin is $-\frac{1}{2}$ we do M $-\frac{1}{2}$ which is done by adding the binary string for $-\frac{1}{2}$ which is the all 1's string, 111. This time the very first QAdd does not trigger and $\ket{s}$ is then added to all of the bits of M using C-NOT gates with carries to check for overflow. 

The Unitary is then performed on $\ket{S}$ depending on the values of the newly calculated M' and J registers using $R_y(\theta_{J,m'})$ \autoref{eq:rotmatrix}. The J register is then updated to J' by adding the value of $\ket{P}$ to J using the QAdd sequence of gates.

To add the second qubit in the values of J' and M' are passed in as the initial register values. It is easy to extend this to many qubits being streamed in one at a time by carefully conditioning the controls on the unitaries, in the general case you need at most N controls for coupling up to N qubits in one at a time. The circuit written here has redundancy in the Identity and ZX gates appearing twice which is shown in the circuit for completeness\autoref{cir:genstreaming}. 
registers.


\begin{table}[H]
\begin{tabular}{ |c c c|c| } 
\hline
 $J_2$ &$J_1$ &$J_0$ &J \\
 \hline
 0 &0 &0 &0 \\ 
 0 &0 &1 &$\frac{1}{2}$ \\ 
 0 &1 &0 &1 \\ 
 0 &1 &1 &$\frac{3}{2}$ \\ 
 \hline 
 1 &0 &0 &-2 \\ 
 1 &0 &1 &-$\frac{3}{2}$ \\ 
 1 &1 &0 &-1 \\ 
 1 &1 &1 &-$\frac{1}{2}$ \\  
 \hline 
\end{tabular}
\quad
\begin{tabular}{ |c c c|c| } 
\hline
 $m_2$ &$m_1$ &$m_0$ &M \\
 \hline
 0 &0 &0 &0 \\ 
 0 &0 &1 &$\frac{1}{2}$ \\ 
 0 &1 &0 &1 \\ 
 0 &1 &1 &$\frac{3}{2}$ \\ 
 \hline 
 1 &0 &0 &-2 \\ 
 1 &0 &1 &-$\frac{3}{2}$ \\ 
 1 &1 &0 &-1 \\ 
 1 &1 &1 &-$\frac{1}{2}$ \\  
 \hline
\end{tabular}
\caption{Tables giving binary Two's complement encoding to spin values of the M and J registers}
\label{fig:encoding}
\end{table}


\newpage
%%%%%%%%%%%%%%%%%%%%%%%%%%%%%% Circuit 5

\begin{figure}[H]
\begin{align}
\Qcircuit @C=0.15cm @R=0.7cm {
%%%%%%%%%%%%%%%%%J-Carry
%j-C2
&\lstick{C_2 \ket{0}} 
&\qw&\qw&\qw&\qw&\qw&\qw&\qw&\qw&\qw&\qw&\qw&\qw&\qw&\qw&\qw&\qw&\qw&\qw&\qw&\qw&\qw&\qw&\qw&\qw&\qw&\qw&\qw&\qw&\qw&\qw&\qw&\qw&\qw&\qw&\qw&\qw&\qw&\qw&\qw&\qw&\qw&\qw&\qw&\qw&\qw&\qw&\qw&\qw&\qw
&\targ&\targ&\targ &\qw
&\ctrl{4} 
&\qw&\qw&\qw &\ctrl{3} &\qw&\qw &\rstick{C_2} &\\
%j-C1
&\lstick{C_1 \ket{0}} 
&\qw&\qw&\qw&\qw&\qw&\qw&\qw&\qw&\qw&\qw&\qw&\qw&\qw&\qw&\qw&\qw&\qw&\qw&\qw&\qw&\qw&\qw&\qw&\qw&\qw&\qw&\qw&\qw&\qw&\qw&\qw&\qw&\qw&\qw&\qw 
&\targ&\targ&\targ &\qw&\qw&\qw&\qw 
&\targ&\targ&\targ &\qw&\qw&\qw&\qw&\qw 
&\ctrl{-1} &\ctrl{-1} &\qw&\qw&\qw&\qw&\qw&\qw&\qw&\qw &\rstick{C_1} &\\
%j-C0
&\lstick{C_0 \ket{0}} \gategroup{1}{2}{3}{3}{1em}{\{}
&\qw&\qw&\qw&\qw&\qw&\qw&\qw&\qw&\qw&\qw&\qw&\qw&\qw&\qw&\qw&\qw&\qw&\qw&\qw&\qw&\qw&\qw&\qw&\qw&\qw&\qw&\qw&\qw&\qw&\qw&\qw&\qw&\qw&\qw&\qw&\qw
&\ctrl{-1} &\ctrl{-1} &\qw &\ctrl{3} &\qw&\qw
&\qw &\ctrl{-1} &\ctrl{-1} &\qw &\ctrl{3} 
&\qw&\qw&\qw&\qw&\qw&\qw&\qw&\qw&\qw&\qw&\qw&\qw&\qw &\rstick{C_0} &\\
%%%%%%%%%%%%%%%%%% J
%J2
&\lstick{J_2 \ket{0}}
&\qw&\qw&\qw&\qw&\qw&\qw&\qw&\qw&\qw&\qw&\qw&\qw&\qw&\qw&\qw&\qw&\qw&\qw&\qw&\qw&\qw&\qw&\qw&\qw&\qw&\qw&\qw&\qw&\qw&\qw&\qw&\qw&\qw&\qw&\qw&\qw&\qw&\qw&\qw&\qw&\qw&\qw&\qw&\qw&\qw&\qw&\qw&\qw&\qw&\qw&\qw&\qw&\qw&\qw&\qw&\qw
&\targ &\targ &\qw&\qw  &\rstick{J'_2} & \\
%J1
&\lstick{J_1 \ket{0}}
&\qw&\qw&\qw&\qw&\qw&\qw&\qw&\qw&\qw&\qw&\qw&\qw&\qw&\qw&\qw&\qw&\qw&\qw&\qw&\qw&\qw&\qw&\qw&\qw&\qw&\qw&\qw&\qw&\qw&\qw&\qw&\qw&\qw&\qw&\qw&\qw&\qw&\qw&\qw&\qw&\qw&\qw&\qw&\qw&\qw&\qw&\qw&\qw&\qw
&\ctrl{-4} &\ctrl{-3} &\qw
&\targ &\targ &\qw&\qw&\qw&\qw&\qw&\qw  &\rstick{J'_1} & \\
%J0
&\lstick{J_0 \ket{0}} \gategroup{4}{2}{6}{3}{1em}{\{}
&\qw&\qw&\qw&\qw&\qw&\qw&\qw&\qw&\qw&\qw&\qw&\qw&\qw&\qw&\qw&\qw&\qw&\qw&\qw&\qw&\qw&\qw&\qw&\qw&\qw 
&\ctrlo{4} &\ctrlo{4} &\qw &\qw&\qw
&\ctrl{5} &\ctrl{4} &\ctrl{4} &\qw&\qw
&\ctrl{-4} &\ctrl{-3} &\qw &\targ &\targ &\qw&\qw 
&\ctrl{-4} &\ctrl{-3} &\ctrl{-3}
&\targ &\targ &\qw&\qw&\qw&\qw&\qw&\qw&\qw&\qw&\qw&\qw&\qw&\qw&\qw
&\rstick{J'_0} \gategroup{4}{60}{6}{61}{1.5em}{\}} &  \\
%%%%%%%%%%%%%%%%%%M-Carry
%m-C2
&\lstick{C_2 \ket{0}} 
&\qw&\qw&\qw&\qw&\qw&\qw&\qw&\qw&\qw&\qw&\qw&\qw&\qw&\qw 
&\targ &\targ &\targ &\qw&\qw&\qw&\qw
&\qw &\ctrl{3} 
&\qw&\qw&\qw&\qw&\qw&\qw&\qw&\qw&\qw&\qw&\qw&\qw&\qw&\qw&\qw&\qw&\qw&\qw&\qw&\qw&\qw&\qw&\qw&\qw&\qw&\qw&\qw&\qw&\qw&\qw&\qw&\qw&\qw&\qw&\qw&\qw&\qw     &\rstick{C_2}    \\
%m-C1
&\lstick{C_1 \ket{0}} 
&\qw &\targ &\targ &\targ &\qw&\qw&\qw 
&\targ &\targ &\targ &\qw&\qw&\qw&\qw
&\qw &\ctrl{-1} &\ctrl{-1} &\qw &\ctrl{3} 
&\qw&\qw&\qw&\qw&\qw&\qw&\qw&\qw&\qw&\qw&\qw&\qw&\qw&\qw&\qw&\qw&\qw&\qw&\qw&\qw&\qw&\qw&\qw&\qw&\qw&\qw&\qw&\qw&\qw&\qw&\qw&\qw&\qw&\qw&\qw&\qw&\qw&\qw&\qw&\qw&\qw &\rstick{C_1}           \\
%m-C0
&\lstick{C_0 \ket{0}} \gategroup{7}{2}{9}{3}{1em}{\{}
&\qw&\qw &\ctrl{-1} &\ctrl{-1} &\qw &\ctrl{3} &\qw&\qw 
&\ctrl{-1} &\ctrl{-1} &\qw &\ctrl{3} &\qw
&\qw&\qw&\qw&\qw&\qw&\qw&\qw&\qw&\qw&\qw&\qw&\qw&\qw&\qw&\qw&\qw&\qw&\qw&\qw&\qw&\qw&\qw&\qw&\qw&\qw&\qw&\qw&\qw&\qw&\qw&\qw&\qw&\qw&\qw&\qw&\qw&\qw&\qw&\qw&\qw&\qw&\qw&\qw&\qw&\qw&\qw&\qw  &\rstick{C_0}     \\
%%%%%%%%%%%%%%%%% M
%M2
&\lstick{m_2 \ket{0}}
&\qw&\qw&\qw&\qw&\qw&\qw&\qw&\qw&\qw&\qw&\qw&\qw&\qw&\qw&\qw&\qw&\qw&\qw&\qw&\qw&\qw 
&\targ &\targ &\qw&\qw
&\ctrlo{3} &\ctrl{3} &\qw&\qw&\qw 
&\qw &\ctrlo{1} &\ctrl{3} 
&\qw&\qw&\qw&\qw&\qw&\qw&\qw&\qw&\qw&\qw&\qw&\qw&\qw&\qw&\qw&\qw&\qw&\qw&\qw&\qw&\qw&\qw&\qw&\qw&\qw&\qw&\qw &\rstick{m'_2}   &    \\
%m1
&\lstick{m_1 \ket{0}}
&\qw&\qw&\qw&\qw&\qw&\qw
&\qw&\qw&\qw&\qw&\qw &\qw&\qw&\qw
&\ctrl{-4} &\ctrl{-3} &\qw &\targ &\targ
&\qw&\qw&\qw&\qw&\qw&\qw&\qw &\qw&\qw&\qw&\qw
&\ctrl{2} &\ctrlo{2} &\qw&\qw&\qw&\qw
&\qw&\qw&\qw&\qw&\qw&\qw&\qw&\qw&\qw&\qw&\qw&\qw&\qw&\qw&\qw&\qw&\qw&\qw&\qw&\qw&\qw&\qw&\qw&\qw  &\rstick{m'_1}   &      \\
%m0
&\lstick{m_0 \ket{0}} \gategroup{10}{2}{12}{3}{1em}{\{} \gategroup{7}{10}{13}{15}{1em}{--} &\qw
&\ctrl{-4} &\ctrl{-3} &\qw &\targ &\targ &\qw
&\ctrl{-4} &\ctrl{-3} &\qw &\targ &\targ &\qw&\qw
&\qw&\qw&\qw&\qw&\qw&\qw&\qw&\qw&\qw&\qw&\qw&\qw&\qw&\qw&\qw&\qw&\qw
&\qw &\qw&\qw&\qw&\qw&\qw&\qw&\qw&\qw&\qw&\qw&\qw
&\qw&\qw&\qw&\qw&\qw&\qw&\qw&\qw&\qw&\qw&\qw&\qw&\qw&\qw&\qw&\qw&\qw &\rstick{m'_0} \gategroup{10}{60}{12}{61}{1.5em}{\}} &&      \\
%%%%%%%%%%%%%%%%%%%%%%%%%%
%Qubit
&\lstick{Spin \ket{s}} &\qw 
&\ctrlo{-1} &\qw &\ctrlo{-4} &\ctrlo{-1} &\qw&\qw
&\ctrl{-1} &\qw &\ctrl{-4} &\ctrl{-1} &\qw&\qw&\qw 
&\ctrl{-2} &\qw &\ctrl{-5} &\ctrl{-2} &\qw&\qw&\qw 
&\ctrl{-3} &\qw&\qw&\qw &\gate{I} &\gate{ZX} \gategroup{6}{27}{13}{30}{1em}{--} 
&\qw&\qw&\qw &\gate{I} &\gate{HX} &\gate{ZX} \gategroup{6}{32}{13}{36}{1em}{--} &\qw&\qw 
&\ctrlo{-7} &\qw &\ctrlo{-10} &\ctrlo{-7} &\qw&\qw&\qw 
&\ctrl{-7} &\qw &\ctrl{-7} &\ctrl{-7} &\qw&\qw&\qw  \gategroup{2}{45}{13}{50}{1em}{--} 
&\ctrl{-8} &\qw &\ctrl{-11} &\ctrl{-8} &\qw&\qw&\qw
&\ctrl{-9} \qw&\qw&\qw&\qw  &\rstick{\ket{P}}         \\ 
%Labels
&&&&&&&&&&& \mbox{Quantum Adder} &&&&&&&&&&&&&&&& \mbox{$1^{st}$ Spin} &&&&&& \mbox{$2^{nd}$ Spin} &&&&&&&&&&&&& \mbox{Quantum Adder} &&&&&&&&&&&&&&&&&&&&&&&&&&&&&&&&&
}
\end{align}
\caption{General streaming circuit}
\label{cir:genstreaming}
\end{figure}
%%%%%%%%%%%%%%%%%%%%%%%%%%% Circuit 6 

\newpage

This can be simplified, removing all of the carries as we take advantage of the fact that as only one qubit is coupled at a time the registers will either always be incremented by $\pm 1$. This enables us to rewrite the circuit in this way and effectively use the $m_1$ qubit as a temporary carry. Here we have also expanded all of the multiple control gates into single control gates. 
%%%%%%%%%%%%%%%%%%%%%%%% Circuit 4

\begin{figure}[H]
\begin{align}
\Qcircuit @C=0.28cm @R=0.7cm {
%%%%%%%%%%%%%%%%%% J
%J2
&\lstick{J_2 \ket{0}} &\qw&\qw&\qw&\qw&\qw&\qw&\qw&\qw&\qw&\qw&\qw&\qw&\qw&\qw
&\qw&\qw&\qw&\qw&\qw&\qw&\qw&\qw&\qw
&\qw &\gate{H} &\gate{V} &\qw &\gate{V^{\dag}} &\qw &\gate{V} &\gate{H} &\qw&\qw
&\qw&\qw&\qw  &\rstick{J'_2}
 & \\
%J1
&\lstick{J_1 \ket{0}} &\qw&\qw&\qw&\qw&\qw&\qw&\qw&\qw&\qw&\qw&\qw&\qw&\qw&\qw
&\qw&\qw&\qw&\qw&\qw&\qw&\qw&\qw&\qw
&\targ &\gate{X} &\ctrl{-1} &\targ &\ctrl{-1} &\targ &\qw &\gate{X} &\qw &\targ &\qw
&\qw&\qw &\rstick{J'_1} && \\
%J0
&\lstick{J_0 \ket{0}} &\qw&\qw&\qw&\qw&\qw&\qw&\qw&\qw&\qw&\qw&\qw&\qw&\qw&\qw
&\qw&\qw&\qw&\qw&\qw&\qw&\qw&\qw&\qw 
&\ctrl{-1} &\targ &\qw &\ctrl{-1} &\qw &\ctrl{-1} &\ctrl{-2} &\targ &\qw&\qw &\gate{X}
&\qw&\qw &\rstick{J'_0}
\gategroup{1}{2}{3}{3}{1em}{\{}
\gategroup{1}{36}{3}{38}{1.5em}{\}}
\gategroup{1}{26}{7}{36}{2.5em}{--} 
&  \\
%%%%%%%%%%%%%%%%% M
%M2
&\lstick{m_2 \ket{0}} &\qw&\qw
&\qw &\gate{H} &\gate{V} &\qw &\gate{V^{\dag}} &\qw &\gate{V} &\gate{H} &\qw&\qw&\qw&\qw
&\ctrl{3} &\qw&\qw&\qw&\qw&\qw&\qw&\qw&\qw&\qw&\qw&\qw&\qw&\qw&\qw 
&\qw&\qw&\qw&\qw&\qw&\qw&\qw  &\rstick{m'_2} & \\
%m1
&\lstick{m_1 \ket{0}} &\qw&\qw
&\targ &\gate{X} &\ctrl{-1} &\targ &\ctrl{-1} &\targ &\qw &\gate{X} &\qw &\targ &\qw&\qw&\qw
&\qw&\qw &\ctrlo{1} &\qw &\ctrlo{1} &\ctrlo{2} &\qw&\qw&\qw&\qw&\qw&\qw&\qw&\qw&\qw&\qw&\qw&\qw&\qw&\qw&\qw  &\rstick{m'_1} & \\
%m0
&\lstick{m_0 \ket{0}} &\qw&\qw 
&\ctrl{-1} &\targ &\qw &\ctrl{-1} &\qw &\ctrl{-1} &\qw &\targ &\qw&\qw &\gate{X} &\qw
&\qw&\qw &\ctrlo{1} &\targ &\ctrlo{1} &\targ &\qw&\qw&\qw&\qw&\qw&\qw&\qw
&\qw&\qw&\qw&\qw&\qw&\qw&\qw&\qw&\qw &\rstick{m'_0} 
\gategroup{4}{2}{6}{3}{1em}{\{} 
\gategroup{4}{36}{6}{38}{1.5em}{\}} 
\gategroup{4}{5}{7}{15}{2.5em}{--} 
&&      \\
%%%%%%%%%%%%%%%%%%%%%%%%%%
%Qubit
&\lstick{Spin \ket{s}} &\qw&\qw 
&\qw &\ctrl{-1} &\qw&\qw&\qw&\qw &\ctrl{-3} &\ctrl{-1} &\qw &\ctrl{-2} &\qw&\qw
&\gate{ZX} &\qw &\gate{W} &\qw &\gate{W^{\dag}} &\qw &\gate{W} 
&\qw&\qw&\qw &\ctrl{-4} &\qw&\qw&\qw&\qw&\qw &\ctrl{-4} &\qw &\ctrl{-5} &\qw&\qw&\qw &\rstick{\ket{P}}       
\gategroup{4}{17}{7}{23}{2em}{--} 
%\gategroup{6}{32}{13}{36}{1em}{--}  
%\gategroup{2}{45}{13}{50}{1em}{--}
\\ 
%Labels
&&&&&&&& \mbox{Quantum Adder on m} &&&&&&&&&&& \mbox{Transform spin} &&&&&&&&&&& \mbox{Quantum Adder on J} &&&&&&&&&&&&&&&
}
\end{align}
\caption{Temporal multiplexed streaming}
\label{cir:tempmultistream}
\end{figure}

\newpage
Where $V$ is the phase gate, 
$ V = \begin{bmatrix}
1 & 0 \\
0 & i \\
\end{bmatrix} $
, $V^{\dag}V = I$ and $V^2 = Z$. $V$ is used here to expand the double controlled Toffoli gate into single control gates in the quantum adder subroutine.

The $W$ gate, $W^2 = HX$ with $W^{\dag} = I$, is used to expand the $HX$ gate into single control gates in the spin transform region. 

The circuit checks that if ($m_1$ XNOR $m_0$) AND ($m_0$ XOR $S_0$) and will then change $m_2$. Then $m_1$ is updated using $m_1 = m_0$ XOR $S_0$. $m_0$ is always incremented by 1, if $\ket{S} = \ket{0}$ increment only $m_0$ by 1 corresponding to adding $\frac{1}{2}$ to the $M$ register. $\ket{S} = \ket{1}$ corresponds to subtracting $\frac{1}{2}$ from the $M$ register by adding the string 111 bitwise to $M$. 
 
For the most positive values of $M$ the Identity is performed on the spin corresponding to the strings $M=001 (J=\frac{1}{2}, M' = \frac{1}{2})$ for the first spin and $M=010 (J=1, M'=1) $ for the second coupled in spins. 

The most negative values of $M$ performs $XZ \ket{S}$ corresponding to the strings $M=111 (J=\frac{1}{2}, M'=-\frac{1}{2})$ for the first spin and $M=110 (J=1, M'=-1)$ for the second spin.

If $M=000 (J=0, M'=0)$ do $XH \ket{S}$ \autoref{cir:tempmultistream}.  

\section{Spatial multiplexing}

The registered circuit, which takes in two qubits can be constructed from 11 gates if the m register is not compressed. The spatially multiplexed minimal gate explicit J \& M 2 qubit transform is shown in \autoref{cir:minspatialmult} which contains 11 two-qubit gates.
Another CNOT can be added to compress the M' register, which means using the same values for M'=0, see \autoref{cir:spatialmulti2} for the 12 two-qubit gate circuit.


\newpage
%%%%%%%%%%%%%%%%%%%%%%%%%% Circuit 3

\begin{figure}[H]
\begin{align}
\Qcircuit @C=0.7cm @R=0.7cm {
%J
&\lstick{J} &\qw&\qw&\qw&\qw&\qw&\qw&\qw&\qw&\qw&\qw&\qw&\qw&\qw &\targ &\qw&\qw  &\rstick{J': \ket{0}=0, \ket{1}=1}\\
%M1
&\lstick{m_1} &\qw&\qw&\qw &\targ &\ctrl{1} &\qw&\qw&\qw &\ctrl{1} &\qw &\ctrl{1} &\ctrl{3} &\qw&\qw&\qw&\qw &\rstick{S_1} &&&&& \mbox{M'}\\
%M0
&\lstick{m_0} &\targ &\ctrl{1} &\qw&\qw &\targ &\ctrl{2} &\gate{X} &\ctrl{2} &\targ &\ctrl{2} &\targ &\qw&\qw&\qw&\qw&\qw &\rstick{\overline{S_0+S_1}} 
\gategroup{2}{20}{3}{22}{1em}{\}} &&&&& \mbox{Register} \\
%S0
&\lstick{\ket{S_0}} &\ctrl{-1} &\gate{ZX} &\qw&\qw&\qw&\qw&\qw&\qw&\qw&\qw&\qw&\qw&\qw&\qw&\qw&\qw &\rstick{\ket{P_0}} 
\gategroup{3}{3}{4}{4}{2em}{--} \\
%S1
&\lstick{\ket{S_1}} &\qw&\qw&\qw  &\ctrl{-3} &\qw &\gate{HX} &\qw &\gate{T} &\qw &\gate{T^{\dag}} &\qw &\gate{T} &\qw &\ctrlo{-4} &\qw&\qw &\rstick{\ket{P_1}}
\gategroup{4}{10}{5}{14}{1em}{_\}} 
\gategroup{2}{10}{5}{14}{1em}{--} \\
%label
&&& \mbox{$1^{st}$ Spin} &&&&&&&& \mbox{ZX gate}
}
\end{align}
\caption{Minimal gate spatial multiplexing}
\label{cir:minspatialmult}
\end{figure}

$HX$ gate triggers if $M=0$ meaning $S_0\neq S_1$ which is implemented using an XOR between $S_0$ \& $S_1$. The other gate ($T$) is triggered when $M=-1$ meaning $S_0=S_1=1$ which is done using an AND (Toffoli) gate  between, $S_0=S_1$ AND $S_1=1$ which is decomposed into 5 two-qubit gates. $T^2$ is the ZX gate, meaning $T^2\ket{S} = XZ\ket{S}$.

The registered circuit, which takes in two qubits can be constructed from 11 gates if the m register is not compressed. The spatially multiplexed minimal gate explicit J \& M 2 qubit transform is shown in \autoref{cir:minspatialmult} which contains 11 two-qubit gates.

\begin{table}[H]
\begin{tabular}{ |c c|c c|c| } 
\hline
 Spin values &&Circuit output &&M value \\
\hline
 $S_1$ &$S_0$ &$S_1$ &$\overline{S_0+S_1}$ &M \\
\hline
 0 &0 &0 &1 &M=+1 \\ 
 0 &1 &0 &0 &M=0 \\ 
 1 &0 &1 &0 &M=0 \\ 
 1 &1 &1 &1 &M=-1 \\ 
\hline 
\end{tabular}
\caption{Table giving M register decoding for minimal gate number}
\label{fig:tab}
\end{table}

\newpage
%%%%%%%%%%%%%%%%%%%%%%% Circuit 2
\begin{figure}[H]
\begin{align}
\Qcircuit @C=0.52cm @R=0.6cm {
%J
&\lstick{J} &\qw&\qw&\qw&\qw&\qw&\qw&\qw&\qw&\qw&\qw&\qw&\qw&\qw&\qw&\qw&\qw&\qw &\targ &\qw&\qw&\qw &\rstick{J': \ket{0}=0, \ket{1}=1}\\
%M1
&\lstick{m_1} &\qw &\gate{H} &\gate{V} &\qw &\gate{V^{\dag}} &\qw &\gate{V} &\gate{H}
&\qw &\qw&\qw&\qw&\qw&\qw &\ctrl{3} &\qw &\ctrlo{1} &\qw&\qw&\qw&\qw &\rstick{S_0 \wedge S_1} &&&&&&& \mbox{M'}\\
%M0
&\lstick{m_0} &\qw&\qw&\qw&\qw&\qw&\qw&\qw&\qw&\qw 
&\targ &\ctrl{1} &\qw &\targ &\ctrl{2} &\qw&\qw &\targ &\qw&\qw&\qw&\qw &\rstick{S_0 \oplus S_1 \oplus \overline{S_0 \wedge S_1}}  
\gategroup{2}{30}{3}{30}{1em}{\}} 
&&&&&&&& \mbox{Register} &&&&& \\
%S0
&\lstick{\ket{S_0}} &\qw&\qw &\ctrl{-2} &\targ &\ctrl{-2} &\targ &\qw&\qw&\qw &\ctrl{-1} &\gate{ZX} 
&\qw&\qw&\qw&\qw&\qw&\qw&\qw&\qw&\qw&\qw &\rstick{\ket{P_0}} 
\gategroup{3}{12}{4}{13}{2em}{--} 
\\
%S1
&\lstick{\ket{S_1}} &\qw&\qw&\qw &\ctrl{-1} &\qw &\ctrl{-1} 
&\ctrl{-3} &\qw&\qw&\qw&\qw&\qw
&\ctrl{-2} &\gate{HX} &\gate{ZX} &\qw&\qw &\ctrlo{-4} &\qw&\qw&\qw &\rstick{\ket{P_1}}
\gategroup{5}{4}{5}{10}{1em}{_\}} 
\gategroup{2}{15}{5}{17}{1em}{--} 
\\
%label
&&&&&& \mbox{$S_0$ AND $S_1$} &&&&&& \mbox{$1^{st}$ Spin} &&& \mbox{$2^{nd}$ Spin}
}
\end{align}
\caption{Spatial multiplexed 2 qubit}
\label{cir:spatialmulti2}
%\vspace{-1cm}
\end{figure}


Another CNOT can be added to compress the M' register, which means using the same values for M'=0, see \autoref{cir:spatialmulti2} for the 12 two-qubit gate circuit.

\begin{table}[H]
\begin{tabular}{ |c c|c c|c| } 
\hline
Spin values &&Circuit output &&M value \\
\hline
 $S_1$ &$S_0$ &$m_1$ &$m_0$ &M \\
\hline
 0 &0 &0 &1 &M=+1 \\ 
 0 &1 &0 &0 &M=0 \\ 
 1 &0 &0 &0 &M=0 \\ 
 1 &1 &1 &0 &M=-1 \\ 
\hline 
\end{tabular}
\caption{Table giving M register decoding for 2 qubit spatial multiplexing}
\label{fig:tab1}
\end{table}


