\documentclass[12pt]{article}

%%%%%%%%%%%%%%%%%%%%%%%%%%%%%%%  Packages  %%%%%%%%%%%%%
\usepackage{amsmath} 
\usepackage{mathtools}
\usepackage{physics}
\usepackage{amssymb}
\usepackage{mathptmx}
\usepackage{array}

%%%%%%%%% FIGurES %%%%%%%%%%%%%%%%%%%%%%%%
\usepackage{textcomp}
\usepackage{graphicx}
\usepackage{caption} 
\usepackage{subcaption}
\usepackage{rotating}
\usepackage{scrextend}
\usepackage{float}

%\graphicspath{ {figures/} }
\usepackage{hyperref}
\hypersetup{colorlinks=true, citecolor=blue, linkcolor=blue}
\renewcommand{\equationautorefname}{Eq.}%
\renewcommand{\figureautorefname}{Fig.}%

%%%%%%%%%%%% LaNgUaGe %%%%%%%%%%%%%%%%%%
\usepackage[latin1]{inputenc}
\usepackage{verbatim}
\usepackage{natbib}
\usepackage{geometry}
\usepackage{qcircuit}
\usepackage{afterpage}
\usepackage{multicol}

%%%%%%%%%%%%%%%%%%%%%%%%%%%%%%%%%%%%%%%%%%%%%%%%%%%%%%%%%
\newgeometry{left=0.8in,right=0.8in,top=1in,bottom=1in}
%%%%%%%%%%%%%%%%%%%%%%%%% EnD oF pAcKaGeS %%%%%%%%%%%%%%%%

\begin{document} %WOOP WOOP!
% Title page 

\title{Nonlinear Optics}
    \author{Oliver Thomas \\[0.5em] \\ Quantum Engineering CDT \\ University of Bristol}
    \date{\today}
    \maketitle

%%%%%%%%%%%%%%%%%%%%%%%%%%%%%%%
    \section{Introduction}

Nonlinear optics,


All current schemes for linear optical quantum computing rely on nonlinear effects at some point. Discrete variable computation requires single photons, a nonlinear effect is used to generate single photons. Either using heralding on a parametric process such as Spontaneous Parametric Down Conversion (SPDC) or Spontaneous Four Wave Mixing (SFWM). Non parametric nonlinear processes such as the light-matter interaction in quantum dots can also be used as a single photon source. There is no deterministic linear single photon source. 



CV computation?




%%%%%%%%%%%%
\bibliographystyle{unsrt}
\bibliography{references}

\end{document}
