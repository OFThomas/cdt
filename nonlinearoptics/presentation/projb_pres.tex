\documentclass{beamer}
%%%%%%%%%%%%%%%%%%%%%%%%%%%%%%%  Packages  %%%%%%%%%%%%%
\usepackage{amsmath} 
\usepackage{mathtools}
\usepackage{physics}
\usepackage{amssymb}
\usepackage{mathptmx}
\usepackage{array}
  
\usepackage[sort&compress]{natbib}     % bib

%%%%%%%%% FIGurES %%%%%%%%%%%%%%%%%%%%%%%%
\usepackage{textcomp}
\usepackage{graphicx}
\usepackage{caption} 
\usepackage{subcaption}
\usepackage{scrextend}
\usepackage{rotating}
\usepackage{float}
\usepackage{hyperref}

\usepackage{multimedia}

%\graphicspath{{./figures/}}
\hypersetup{colorlinks=true, citecolor=blue, linkcolor=blue}
\renewcommand{\equationautorefname}{Eq.}
\renewcommand{\figureautorefname}{Fig.}
 
%%%%%%%%%%%% LaNgUaGe %%%%%%%%%%%%%%%%%%
\usepackage{verbatim}
\usepackage{natbib}
\usepackage{wrapfig}
\usepackage[utf8]{inputenc}

%%%%%%%%%%%%%% PhYsIcS %%%%%%%%%%%%%%%%%%%%%%%

\renewcommand{\annia}{\hat{a}}
\renewcommand{\annib}{\hat{b}}
\renewcommand{\creata}{\hat{a}^\dagger}
\renewcommand{\creatb}{\hat{b}^\dagger}

\renewcommand{\a}{a^ }
\renewcommand{\b}{b^ }
\renewcommand{\adag}{a^\dagger}
\renewcommand{\bdag}{b^\dagger}

\usepackage{qcircuit}



%%%%%%%%%%%%%%%%%5 TIKZ %%%%%%%%%%%%%%%%%%%%

\usepackage{tikz}
\usetikzlibrary{positioning,calc}

%\tikzset{>=stealth}

%\newcommand{\tikzmark}[1]{\tikz[baseline,remember picture] \node[anchor=base, #1] (#1) {};}
%\newcommand{\tikzmark}[3][]{\tikz[overlay,remember picture,baseline] \node [anchor=base,#1](#2) {#3};}
\newcommand{\tikzmark}[3][]{\tikz[remember picture,baseline] \node [anchor=base,#1](#2) {$#3$};}
\usetheme{PaloAlto}

\title{A Scalable Method for Modelling Nonlinear optics with the Bloch-Messiah reduction}
\author{Oliver Thomas, Dara McCutcheon, Will McCutcheon}
\institute{Quantum Engineering CDT \\ University of Bristol}
\date{\today}

% plan

\begin{document}

% slide 1
\frame{\titlepage}

% slide 2
\begin{frame}
\frametitle{Overview}
\begin{itemize}
    \item What is nonlinear optics?
    \item Why do we care about it?
    \item Gaussian optics
    \item What I have been doing
    \item Outlook
\end{itemize}
\end{frame}

%slide 3
\begin{frame}
\frametitle{Motivation quantum nonlinear optics}
\begin{columns}
\column{0.5\textwidth}
    \begin{block}{The good}
    Spontaneous Parametric processes, SPDC, SFWM
    \begin{itemize}
        \item Heralded single photon sources
        \item Entangled photon pair generation (polarisation, spatial)
    \end{itemize}
    Kerr processes 
    \begin{itemize}
        \item Self-Phase modulation (SPM), generating Bannana states (CV)
        \item Cross-Phase modulation (XPM) for sensing
    \end{itemize}
    \end{block}
%
\column{0.5\textwidth}
    \begin{block}{The bad}
        Spontaneous parametric processes
        \begin{itemize}
            \item Generating more than two photons -$>$ bad for quantum computing
            \item Understanding filtering
        \end{itemize}
        All Kerr nonlinear processes 
        \begin{itemize}
            \item SPM -$>$ Spectral broadening
            \item XPM -$>$ Unwanted phase shifts on single photons due to propagation of the pump 
        \end{itemize}
        \end{block}
\end{columns}

\end{frame}

% slide 4
\begin{frame}
\frametitle{What do we mean by nonlinear optics?}
\begin{itemize} 
    \item Roughly processes that conserve energy but do not conserve photon number. 
        \begin{equation}
            \vec{P}= \chi^{(1)} \vec{E_1} + \chi^{(2)}\vec{E_1}\vec{E_2} + \chi^{(3)}\vec{E_1}\vec{E_2}\vec{E_3} + \dots
        \end{equation}
\end{itemize}
Here we are going to talk about squeezing, i.e SPDC or SFWM, Hamiltonians are then of the form, 
\begin{equation} 
    \hat{H} = A \creata_S \creata_I \annia_P + h.c.
\end{equation}
\begin{equation} 
    \hat{H} = A \creata_S \creata_I \annia_P \annia_P + h.c.
\end{equation}
%\textbf{Note} for the rest of this presentation I will drop the hat notatiaion and using the convention a, b are annihilation operators in modes a \& b
\end{frame}

%slide 9
\begin{frame}
\frametitle{Gaussian Optics}

    \begin{itemize}
    \item Using the undepleted pump approximation we can write the Hamiltonians as terms which are at most quadratic in creation and annihilation operators. 
\end{itemize}
        \begin{equation}
        \hat{U} = \exp[-\frac{i}{\hbar}\left( \tikzmark{P}{P} \int d\omega_1 \int d\omega_2 \tikzmark{f}{f(\omega_1,\omega_2)} \tikzmark{aa}{ \creata_s(\omega_1) \creata_i(\omega_2)} + h.c. \right) ]
    \end{equation}

    \begin{tikzpicture}[overlay, remember picture, node distance=1cm]
    
    \node[red] (power) [below =of P]{Power};
    \draw[red,->,thick] (power) to [in=-90,out=45] (P);
    
    \node[blue,xshift=0cm] (jsa) [below =of f]{JSA};
    \draw[blue,->,thick] (jsa) to [in=-90,out=135] (f.south);
    
    \node[purple] (sigidler) [below =of aa]{Signal \& Idler};
    \draw[purple,->,thick] (sigidler) to [in=-90,out=45] (aa.south);
\end{tikzpicture}
\vspace{10pt} 
\begin{itemize}
    \item  Just like Beamsplitters can be written as unitary matrices,
\end{itemize} 
\begin{equation}
    \begin{bmatrix} \vec{b} \end{bmatrix}= \textbf{U} \begin{bmatrix}\vec{a}\end{bmatrix}
\end{equation}

\begin{itemize}
    \item We want to extend the type of transforms to all Gaussian transforms\footnote{These are linear symplectic transforms which conveniently can be written as a matrix \cite{adesso2014continuous}} 
\end{itemize}
\vspace{-10pt}
    \begin{equation}
    \begin{bmatrix} 
        \vec{\b}   \\
        \vec{\bdag}
    \end{bmatrix}
    = 
    \textbf{M}
    \begin{bmatrix}
        \vec{\a} \\
        \vec{\adag}
    \end{bmatrix}
\end{equation}

\end{frame}

% slide 10
\begin{frame}
\frametitle{Types of Gaussian transformations}
%
\begin{figure}[h]
\begin{align*}
\centering
\vspace{-20pt}
    \Qcircuit @C=0.6cm @R=0.1cm{
    %1
        &\lstick{a_1} &\qw &\multigate{1}{Squeezer} &\qw &\qw &\measureD{} \\
    %2
        &\lstick{a_2} &\qw &\ghost{Squeezer} &\qw  &\multigate{1}{BS} &\measureD{} \\
    %3
        &\lstick{a_3} &\qw &\multigate{1}{Squeezer} &\qw &\ghost{BS} &\measureD{} \\
    %4
        &\lstick{a_4} &\qw &\ghost{Squeezer} &\qw &\qw &\measureD{} \\
}
\end{align*}
\caption{Two source HOM dip}
\end{figure}
%
    \vspace{-20pt}
% 
\begin{figure}[h]
\begin{align*}
\centering
    \Qcircuit @C=0.5cm @R=0.1cm{
    %1
        &\lstick{a_1} &\qw &\multigate{1}{Squeezer} &\qw &\qw &\measureD{} & \\
    %2
        &\lstick{a_2} &\qw &\ghost{Squeezer} &\qw  &\multigate{1}{PBS} &\gate{\pi/4} &\measureD{H \& V} \\
    %3
        &\lstick{a_3} &\qw &\multigate{1}{Squeezer} &\qw &\ghost{PBS} &\measureD{} \\
    %4
        &\lstick{a_4} &\qw &\ghost{Squeezer} &\qw &\qw &\measureD{} \\
}
\end{align*}
\caption{Type-1 Fusion gate}
\end{figure}
\footnotetext{These are two-mode squeezers}
%
\end{frame}


    % slide 
\begin{frame}{Schmidt decomposition}
    We can re-write the Hamiltonian using a Schmidt-decomposition \cite{lvovsky2007decomposing}as,
    \begin{equation}
    P' F(\omega_1,\omega_2) = \sum_k r_k \psi_k(\omega_1) \phi_k(\omega_2)
    \end{equation}

    Where $r_k$ is the Schmidt number, $ \psi $ \& $\phi $ are unitaries.\newline

    To solve this numerically we discrete the function and the Schmidt-decomposition is then the Singular value decomposition (SVD) of the JSA (F).
    \begin{equation}
        P' \textbf{F}_{(\omega_1, \omega_2)} = \sum_k r_k \textbf{U}_{(\omega_1, k)} \textbf{V}_{(k, \omega_2)}^\dagger
    \end{equation}
    \begin{itemize}
        \item with $ \psi_k(\omega_1) $ is the k-th row and $\omega_1$-th column of $\textbf{U}_{(\omega_1, k)}$,
        \item with $ \phi_k(\omega_2) $ is the $\omega_2$-th row and k-th column of $\textbf{V}^\dagger_{(k,\omega_2)}$
    \end{itemize}
\end{frame}

%slide 7
\begin{frame}{Joint Spectral Amplitudes (JSAs)} 
    \centering
        \begin{figure}
            \movie[height=0.65\textwidth,width=1\textwidth, loop, autostart]{}{jsasepgauss.mp4}
        \end{figure}
        \footnotetext{Moving to the rotating frame...}
    \end{frame}

% slide 
    \begin{frame}{Seperable JSAs Schmidt modes}
        
    \begin{figure}
        \centering
    \begin{subfigure}{0.4\textwidth}
        \includegraphics[width=1\textwidth]{sepschmidtmodesocc.png}
    \end{subfigure}
    ~
    \begin{subfigure}{0.5\textwidth}
        \includegraphics[width=1\textwidth]{sepsingle_sig_idler1.png}
        \caption{Signal (red) and Idler (blue)}
        \end{subfigure}
    \end{figure}

    \begin{equation}
        F(\omega_1,\omega_2) = exp\left[-0.2 \left( \left(\frac{\omega_1}{\sigma_1} \right)^2+\left(\frac{\omega_2}{\sigma_2} \right)^2 \right) \right]
    \end{equation}
    normalised so that,
    \begin{equation}
    \int d\omega_1 \int d\omega_2 F(\omega_1,\omega_2) = 1
    \end{equation}
\end{frame} 

%slide 8
\begin{frame}{Non-separable JSAs} 
    \begin{figure}
        \centering
            \movie[height=0.65\textwidth,width=1\textwidth, loop, autostart]{}{jsanotsepsincgauss.mp4}
        \end{figure}
    \end{frame}
%
    \begin{frame}{Non-separable JSAs Schimdt modes}
    \begin{figure}
        \centering
        \begin{subfigure}{0.4\textwidth}
            \includegraphics[width=1\textwidth]{notsepschmidtmodesocc.png}
        \end{subfigure}
        ~
        \begin{subfigure}{0.5\textwidth}
        \includegraphics[width=1\textwidth]{notsepsingle_sig_idler1.png}
        \caption{Signal (red) and Idler (blue)}
        \end{subfigure}

    \begin{equation}
        F(\omega_1,\omega_2) = sinc(2(\omega_1-\omega_2))exp\left[-0.1 \left( \left(\frac{\omega_1}{\sigma_1} \right)^2+\left(\frac{\omega_2}{\sigma_2} \right)^2 \right) \right]
    \end{equation}
    \vspace{-5pt}
    normalised so that,
    \begin{equation}
    \int d\omega_1 \int d\omega_2 F(\omega_1,\omega_2) = 1
    \end{equation}
    \end{figure}

\end{frame} 

\begin{frame}{Reducing the size of the state-space}
    \begin{itemize}
        \item The Schmidt decomposition lets us represent the system in a finite number of broadband modes ($\psi_k(\omega_1), \phi_k(\omega_2)$) \cite{wasilewski2006pulsed}
        \item Defining new mode operators for signals, $\hat{A}_k$ and idlers, $\hat{B}_k$

    \end{itemize}
    \begin{equation}
    \hat{A}_k = \int d\omega_s \psi_k(\omega_s) \annia_s
    \end{equation}
    \begin{equation}
    \hat{B}_k = \int d\omega_i \phi_k(\omega_i) \annia_i
    \end{equation}
\end{frame}

% slide 10
\begin{frame}
\frametitle{Correlations in a HOM dip}
%

\begin{figure}[h]
\begin{align*}
\centering
    \Qcircuit @C=0.6cm @R=0.2cm{
    %1
        &\lstick{A_{1,k}} &\qw &\multigate{1}{Squeezer 1} &\qw &\qw &\measureD{} \\
    %2
        &\lstick{B_{2,k}} &\qw &\ghost{Squeezer 1} &\qw  &\multigate{1}{BS} &\measureD{} \\
    %3
        &\lstick{B_{3,k}} &\qw &\multigate{1}{Squeezer 2} &\qw &\ghost{BS} &\measureD{} \\
    %4
        &\lstick{A_{4,k}} &\qw &\ghost{Squeezer 2} &\qw &\qw &\measureD{} \\
}
\end{align*}
\caption{Two source HOM dip}

\end{figure}
\end{frame}
%slide 11
\begin{frame}{Two squeezers JSA} 
    \begin{figure}
        \centering
        \begin{subfigure}{0.45\textwidth}
       \movie[height=0.9\textwidth, width=1\textwidth, loop, autostart]{}{jsa12pickme.mp4}
        \end{subfigure}
        ~
        \begin{subfigure}{0.45\textwidth}
        \includegraphics[width=1\textwidth]{single_sig_idler12.png}
        \caption{The signals $A_k$ (red) and idlers $B_k$ (blue)}
    \end{subfigure}
   \end{figure}

\end{frame} 

% slide 12
\begin{frame}
    \frametitle{G(4) correlation function}
    \begin{equation}
        G^{(4)} = \frac{ \left< \creata_1 \creata_2 \creata_3 \creata_4 \annia_1 \annia_2 \annia_3 \annia_4 \right>}
        {\left< \creata_1 \annia_1 \right> \left< \creata_2 \annia_2 \right> \left< \creata_3 \annia_3 \right> \left< \creata_4 \annia_4 \right>}
    \end{equation}
    Where,
    \begin{equation}
        a_i = \sum_j a_{i}(\omega_j)
    \end{equation}
    Meaning we sum over all of the spectral modes of the spatial modes (1,2,3,4) separately.
    We end up with, 
    \begin{equation}
        G^{(4)} = 1 - \left( \frac{2 \mid cosh(r) \mid^2}{\mid cosh(r) \mid ^2 + \mid sinh(r) \mid^2} sin(\theta) cos(\theta) \right)^2
    \end{equation}
\end{frame}

\begin{frame}{G(4) correlation function}

    \begin{figure}[h]
       \movie[height=0.7\textwidth, width=1\textwidth, loop, autostart]{}{g4python.mp4}
    \end{figure}
\end{frame}

\begin{frame}{G(4) correlation function}

    \begin{figure}
    \centering
        \begin{subfigure}{0.52\textwidth}
        \includegraphics[width=\textwidth]{g4plot1.png} 
        \end{subfigure}
        ~
        \begin{subfigure}{0.52\textwidth}
        \includegraphics[width=\textwidth]{g4plot3.png}
        \end{subfigure}

        \end{figure}


\end{frame}

%slide 13
\begin{frame}
\frametitle{Summary}
\begin{itemize}
    \item The Schmidt decomposition is a useful technique to exactly reduce the continuum of spectral modes in a JSA to a discrete set of broadband modes.
    \item We have derived a closed form expression for the $G^{(4)}$ and it agrees with computational simulation for two identical, separable squeezers.
    \end{itemize}
Outlook,
\begin{itemize}
    \item Currently, the method creating the sympectic matrix $\textbf{M}$ involves using the Bloch-Messiah reduction which fully diagonalises it. 
    \item This flattens all of the degrees of freedom and puts them on equal footing which is not the case experimentally!
    \item We want to generalise the method so that it respects spatial and spectral degrees of freedom separately.
\end{itemize}
\end{frame}

\begin{frame}{References}
\bibliographystyle{unsrt}
\bibliography{refs}
\end{frame}
\end{document}
