\documentclass[10pt]{IEEEtran}

%%%%%%%%%%%%%55 IEEE stuff %%%%%%%%%%%%%%
\usepackage[T1]{fontenc} % optional
\usepackage{cite}

%%%%%%%%%%%%% MATHS %%%%%%%%%%%%%%%%%%%%
\usepackage{amsmath}
\interdisplaylinepenalty=2500   % for multiline eqs

\usepackage[cmintegrals]{newtxmath}
\usepackage{bm} % optional

%%%%%%%%%% Packages language %%%%%%%%%%

\usepackage{tcolorbox}  % boxes for minted in figures
\usepackage{verbatim}   % comments
\usepackage{url}        % cite url
\usepackage{footnote}       % footnotes
\makesavenoteenv{tabular}   % stop enviroments breaking footnotes

%%%%%%%%% Figures %%%%%%%%%%%%%%%%%%%%%%%
\usepackage{graphicx} 
\usepackage{caption}    % subfigures
\usepackage{float}      % images
\usepackage{hyperref}   % colourful refs
\usepackage{wrapfig}    % Oli

\begin{document}

\title{Approximations in Quantum Nonlinear Optics }
\author{Oliver Thomas}

\maketitle


\begin{abstract}
    Time ordering effects and solving to zeroth order \\
    Classical pump approximation \\
    JSAs \\
\end{abstract}

Hillery \cite{drummond2014quantum}



\begin{comment}
% fig example
\begin{figure}[!h]
\centering
\includegraphics[width=2.5in]{myfigure.png}
\caption{Simulation results for the network.}
\label{fig_sim}
\end{figure}


Subfigure
\begin{figure*}[!h]
\centering
\subfloat[Case I]{\includegraphics[width=2.5in]{subfigcase1}
\label{fig_first_case}}
\hfil
\subfloat[Case II]{\includegraphics[width=2.5in]{subfigcase2}
\label{fig_second_case}}
\caption{Simulation results for the network.}
\label{fig_sim}
\end{figure*}

table 
\begin{table}[!h]
\renewcommand{\arraystretch}{1.3}
\caption{A Simple Example Table}
\label{table_example}
\centering
\begin{tabular}{c||c}
\hline
\bfseries First & \bfseries Next\\
\hline\hline
1.0 & 2.0\\
\hline
\end{tabular}
\end{table}

\end{comment}


\bibliographystyle{IEEEtran}
\bibliography{IEEEabrv,refs.bib}

\end{document}
